\documentclass{IOS-Book-Article}
%\usepackage{times}
%\usepackage{helvet}
%\usepackage{courier}
\usepackage[table]{xcolor}
\usepackage{graphicx}
\usepackage{amssymb,amsmath,amsthm,amsfonts}
\usepackage[vlined,ruled,linesnumbered,algo2e]{algorithm2e}

\usepackage{algpseudocode}
\usepackage{float}
\usepackage{tikz}
\usepackage{color}
\usepackage{calc}
\usepackage{subfig}

\usepackage{booktabs}
\usepackage{multirow}
\def\hb{\hbox to 10.7 cm{}}


\newcommand{\X}{\mathcal{X}}
\newcommand{\g}{\mathcal{G}}
\newcommand{\var}{\textsf{var}}
\newcommand{\Lit}{{\cal L}}

\newcommand {\SAT}{\textsf{SAT}}
\newcommand{\degr}{\textsf{d}}

\newcommand{\scc}{\textsf{scc}}
\newcommand{\FL}{\textsf{FL}}
\newcommand{\BL}{\textsf{BL}}
\newcommand{\BLap}{{\sf \widehat{BL}}}
\newcommand{\NBLap}{{\sf \overline{BL}_\downharpoonright}}
\newcommand{\BV}{\textsf{BV}}
\newcommand{\Pre}{\textsf{Pre}}

\newcommand{\tool}{{\sc Bone}\xspace}
\newcommand{\minibones}{{\sc Minibones}\xspace}
\newcommand{\NBL}{\overline{\textsf{BL}}}

\newcommand{\cls}{\textsf{cls}}
\newcommand{\cl}{\mathcal{C}}
\newcommand{\f}{\textsf{F}}
\newcommand{\Cnt}{\textsf{Cnt}}
\newcommand{\dens}{\textsf{Dnt}}

\newtheorem{theorem}{Theorem}
\newtheorem{lemma}{Lemma}
\newtheorem{corollary}{Corollary}
\newtheorem{proposition}{Proposition}
\newtheorem{definition}{Definition}
\newtheorem{example}{Example}
\newtheorem{remark}{Remark}

\begin{document}

\pagestyle{headings}
\def\thepage{}

\begin{frontmatter}              % The preamble begins here.


%\pretitle{Pretitle}
\title{Towards Backbone Computing: A Greedy-Whitening Based Approach}

\markboth{}{January 2017\hb}
%\subtitle{Subtitle}

\author[A]{\fnms{Yueling} \snm{Zhang}
% \thanks{Corresponding Author: Book Production Manager, IOS Press, Nieuwe Hemweg 6B,
% 1013 BG Amsterdam, The Netherlands; E-mail:
% bookproduction@iospress.nl.}
},
\author[A]{\fnms{Min} \snm{Zhang}}
,
\author[A]{\fnms{Geguang} \snm{Pu}}
,
\author[B]{\fnms{Fu} \snm{Song}}
,
\author[A]{\fnms{Jianwen} \snm{Li}}

\address[A]{National Research Center of Trustworthy Embedded Software
        \\ East China Normal University, China}
\address[B]{School of Information Science and Technology ShanghaiTech University, China}

\begin{abstract}
%Propositional satisfiability (SAT) is widely used in both theoretical and practical applications in computer science. 
Backbone is the set of literals that are true in all formula's models.  Computing  backbone efficiently is the key to  accelerate SAT solving, which widely used in fault localization, product configuration and formula simplification.
In this paper, we propose a greedy-whitening based approach for computing backbone of propositional Boolean formulae.
The Greedy-based algorithm is constructed to compute an under-approximation of non-backbone. Moreover, we also propose 
a Whitening-based algorithm with two heuristic strategies to compute an approximation set of backbone.
The exact backbone is computed by applying iterative test backbone on the approximations.
We implemented our approach in a tool \tool and conducted experiments on instances from Industrial tracks of SAT Competitions
between 2002 and 2016. For industrial benchmarks, experimental results demonstrate that \tool solves as many 34 formulae as \minibones does (72 formulae in total in 3600 seconds), solves (49 formulae)  2 more formula than \minibones(47 formulae)(72 formulae in total in 16000 seconds).   In the case of 3600 seconds, \tool saved 18\% solving time in total.
\minibones performs better on \textit{mrpp} benchmark (4\% times faster), while \tool performs better on \textit{manthey} benchmark, saving 34\% solving time.
On the other side, we implemented our approach in a tool \tool and conducted experiments on instances from Uniform Random-3-SAT. Experiments with 6606 small formulae generated from 86 unsatisfiable formulae indicate that \tool is able to find more backbone quicker  when the number of variables in the same clause of the formulae is in some range.
\end{abstract}

\begin{keyword}
backbone\sep satisfiability\sep
approximation\sep greedy\sep whitening
\end{keyword}
\end{frontmatter}
\markboth{January 2017\hb}{January 2017\hb}




\section{Introduction}
The \textit{backbone} of a satisfiable formula is a set of literals that are true in all models of the formula,
which plays an important role for understanding the hardness of problems in computation complexity.
For instance, the presence of backbones provides a good explanation for the apparent inevitably high cost of heuristic search near the phase boundary  for satisfiability problem \cite{MZKST99}.

The identification of backbone also has many practical applications. Backbone improves the performance of the random SAT solver like  WalkSAT~\cite{SBK1993} by making biased moves in a local search~\cite{ZWR2003,MAR2007}. For optimization area,  backbone can significantly contribute to the Lin-Kernighan local search algorithms for Travel Salesman Problem~\cite{ZWL2005}. Another recent successful application of backbones is post-silicon fault localisation in integrated circuits~\cite{ZWSM11,ZWM11}.

However, computing backbone is co-NP hard \cite{Jan10}.  Many heuristic approaches were propose to compute backbone in different setting, such as model enumeration, iterative SAT-testing and filtering with modern SAT solvers.
Marques-Silva et al.  conducted an experimental evaluation by integration existing algorithms with optimisations in a modern SAT solver and showed that backbone computation for large practical formulae is feasible \cite{MJML2010,JLMS12,JLM15}.


In this paper, we propose a novel Greedy-Whitening based approach \tool for computing backbones. The insight of this approach has two-folds: 1) we present a fast procedure in to compute an under-approximation non-backbone set, which helps to prune the search space during the computation of backbone; 2) we also construct an approximation of backbone in polynomial time, and the element in this set has high possibility to be a backbone literal.
Finally, we compute the exact backbone using SAT solvers.

We implemented our approach in a tool \tool and evaluated this tool with empirical experiments. We tested 72 satisfiable industrial formulae and automatically generate 6606 formulae from 86 unsatisfiable formulae from SAT competitions\footnote{http://www.satcompetition.org/} during 2002 to 2016. In total, \tool solved 6640 more formulae, 34 from satisfiable industrial formulae and 6606 from generated formulae. \tool reduces 38\% solving time comparing to \minibones in benchmark \textit{mrpp}, and 21\% total solving time.

\section{Preliminaries}\label{sec:prel}

We fix a finite set  $\X$ of \emph{Boolean variables}.
A \emph{literal} $l$ is either a Boolean variable $x\in \X$ or its negation $\neg x$.
The negation of a literal $\neg x$ is $x$, i.e., $\neg\neg x=x$.
A \emph{clause} $\phi$ is a disjunction of literals $\bigvee_{i=1}^n l_i$, which may be regarded as
the set of literals $\{l_i\mid 1\leq i\leq n\}$. W.l.o.g., we assume that for every
clause $\phi$, if $l\in\phi$, then $\neg l\not\in \phi$.

A \emph{formula} $\Phi$ over $\X$ is a Boolean combination of variables $\X$.
We assume that formulae are given in conjunctive normal
form (CNF), namely each formula $\Phi$ is a conjunction of clauses $\bigwedge_{i=1}^n\phi_i$ which may be regarded as a set of clauses $\{\phi_i\mid 1\leq i\leq n\}$. Given a formula $\Phi$, let $\var(\Phi)$ (resp. $\Lit(\Phi)$  and $\cls(\Phi)$) denote the set of variables (resp. literals and clauses) used in $\Phi$.
We use $\|\Phi\|$ to denote $\sum_{\phi\in\Phi}|\phi|$.
We use $\neg\Lit(\Phi)$ to denote the set $\{\neg l \mid l\in \Lit(\Phi)\}$.
The \emph{size} $|\Phi|$ of $\Phi$ is the number of literals of $\Phi$.
Given a formula $\Phi$ and a literal $l\in\Lit(\Phi)$,
let $\Phi_{l}\subseteq \Phi$ be the set of clauses $\{\phi\in\Phi\mid l\in\phi\}$.
Given two different variables $v_1, v_2\in\var(\Phi)$, $v_1$ and $v_2$ are adjacent variables iff there exists a clause $\phi\in\Phi$ such that $(v_1\in\phi)\wedge(v_2\in\phi)$.


An \emph{assignment} is a function $\lambda: \X \rightarrow \{0,1\}$, where $1$ (resp. $0$) denotes true (resp. false).
Given an assignment $\lambda$ and a literal $l$ that is $x$ or $\neg x$, let $\lambda[\neg l]$ be the assignment which is equal to $\lambda$
except for $\lambda[\neg l](x)=\neg \lambda(x)$. Given a set of variables $x=\{x_1,...,x_n\}$, let $\lambda[\neg L]$ denote the assignment
$\lambda[\neg x_1]...[\neg x_n]$.
An assignment $\lambda$ \emph{satisfies} a formula $\Phi$, denoted by $\lambda\models \Phi$, iff assigning $\lambda(x)$ to $x$ for $x\in\var(\Phi)$ makes $\Phi$ true.

\iffalse
An assignment $\lambda$ is a \emph{model} of $\Phi$ if $\lambda\models \Phi$.
Given two models $\lambda_1$ and $\lambda_2$ of $\Phi$, $\lambda_2$ can be \emph{generated} from $\lambda_1$, denoted by $\lambda_1\Rightarrow_\Phi\lambda_2$, if there exists a literal $l$ such that $\lambda_2=\lambda_1[\neg l]$.
Let $\Rightarrow^*_\Phi$ denote the \emph{reflexive transitive} closure of $\Rightarrow_\Phi$.
Formally, for every model $\lambda$ of $\Phi$, $\lambda\Rightarrow^*_\Phi\lambda$ and $\lambda_1\Rightarrow^*_\Phi\lambda_3$ if $\lambda_1\Rightarrow^*_\Phi\lambda_2$ and $\lambda_2\Rightarrow^*_\Phi\lambda_3$.
A \emph{model cluster} $\cl_\Phi(\lambda)$ of $\lambda$ is the set of models $\{\lambda'\mid \lambda\Rightarrow^*_\Phi\lambda'\}$.
A literal $l$ is a \emph{frozen literal} in a model $\lambda$ of $\Phi$,
if $l$ takes the same value in all the models of $\cl_\Phi(\lambda)$. Let $\FL(\Phi,\lambda)$ denote the set of all the frozen literals
in the model $\lambda$ of $\Phi$. %, and $\FL(\Phi)$ denote the intersection of $\FL(\Phi,\lambda)$ for all the models $\lambda$ of $\Phi$.
\fi

A formula $\Phi$ is \emph{satisfiable} iff there exists an assignment $\lambda$ such that $\lambda\models \Phi$.
Given a formula $\Phi$, the \emph{satisfiability problem} is to decide whether $\Phi$ is satisfiable or not.

\smallskip

\begin{definition}[Backbone]
\label{def:backbone}
Given a satisfiable formula $\Phi$, a literal $l$ is a \emph{backbone literal} of $\Phi$ iff for all assignments $\lambda$ such that $\lambda\models\Phi$,
$\lambda\models l$. The \emph{backbone} $\BL(\Phi)$ of $\Phi$ is the set of backbone literals of $\Phi$.
\end{definition}

\iffalse
Since a backbone literal takes same value in all the models and a frozen literal takes same value in a model cluster instead of all the models,
we get that:

\begin{proposition}\label{prop:Frozen-backbone}
For any formula $\Phi$ and model $\lambda$ of $\Phi$, $\BL(\Phi)\subseteq\FL(\Phi,\lambda)$.
\end{proposition}
Computing $\FL(\Phi,\lambda)$ of a given model is in polynomial time. However, the generation of the first model in each $\cl_\Phi(\lambda)$ still needs a SAT testing.
\fi

It is known that the backbone $\BL(\Phi)$ for each formula $\Phi$ is unique \cite{JLM15}.
The backbone of an unsatisfiable formula can be defined as an empty set. Therefore, in this work, we focus on satisfiable formulae.
We will use $\NBL(\Phi)$ to denote the set
$\Lit(\Phi)\setminus \BL(\Phi)$.

\begin{theorem}
\label{thm:co-NP}\cite{Jan10}
Given a satisfiable formula $\Phi$ and a literal $l$, deciding whether $l$ is a backbone literal is co-NP-complete.
\end{theorem}

\begin{definition}[Satisfied literal]
Given a model $\lambda$ of the formula $\Phi$ and a clause $\phi\in\Phi$, for each literal $l\in\phi$, $l$ is a \emph{satisfied literal}
of $\phi$ iff $\lambda\models l$. $l$ is a \emph{unique satisfied literal} of $\phi$ if there is no satisfied literal $l'$ of $\phi\setminus\{l\}$.
\end{definition}

Let us consider the formula $\Phi=\{\neg x_1 \vee \neg x_2, x_1, x_3 \vee x_4\}$,
$\var(\Phi)=\{x_1, x_2, x_3, x_4\}$, $\Lit(\Phi)=\{\neg x_1, x_1, \neg x_2, x_3, x_4\}$, $\Phi_{\neg x_2}=\{\neg x_1 \vee \neg x_2\}$ and $\BL(\Phi)=\{x_1, \neg x_2\}$.
Given a model $\lambda$ such that $\lambda(\neg x_1)=1,\lambda(\neg x_2)=0$. The satisfied literal of clause ($\neg x_1\vee\neg x_2$) is $\neg x_1$.

Given a formula $\Phi$ and a literal $l$, let $\Cnt(\Phi,l)$ denote the number of occurrence of $l$ in $\Phi$.
The \emph{density} $\dens(\Phi,l)$ of $l$ in $\Phi$ is defined as
\[
\dens(\Phi, l)=\frac{\Cnt(\Phi,l)}{\|\Phi_l\|}.
\]

\section{Overview of our Approach}\label{sec:overview}
Figure \ref{flow} presents the overview of our approach \tool, which consists of three components. Taking a satisfiable formula
$\Phi$ as an input, \tool first computes an under-approximation $\NBLap(\Phi)\subseteq \NBL(\Phi)$ of the non-backbone of
$\Phi$. Then, \tool computes an approximation $\BLap(\Phi)$ of the backbone of $\Phi$ based on the set $\NBLap(\Phi)$, where each literal $l\in \BLap(\Phi)$ has a high possibility to be a backbone literal of $\Phi$.
Finally, \tool removes non-backbone literals from $\BLap(\Phi)$ and adds backbone literals into $\BLap(\Phi)$ to compute the exact backbone of $\Phi$.
\begin{figure*}[t]
   \includegraphics[scale=0.75]{Framework}
  \hspace*{30mm} \includegraphics[scale=0.75]{Fig-backbone}
   \caption{Overview of our approach}
   \label{flow}
\end{figure*}

As shown in Figure \ref{flow}, $\NBLap(\Phi)$ only contains a part of non-backbone literals.
Most of the literals in $\BLap(\Phi)$ are backbone, only a small part of them is non-backbone.
In iterative testing (step 3), experiments show that solving time are saved by testing the literals in $\BLap(\Phi)$ first. It's because that there is a higher possibility to choose a backbone from $\BLap(\Phi)$. With more known backbone literals, SAT testings are accelerated.

\medskip
\noindent{\bf Computing an under-approximation of non-backbone.}
Given a satisfiable formula $\Phi$, we first compute a model $\lambda$ of $\Phi$ by calling a SAT solver.
From the model $\lambda$, we compute a base under-approximation of non-backbone.
Later, we apply a Greedy-based algorithm to add more non-backbone literals into the base set, which results in $\NBLap(\Phi)$.

The algorithm iteratively computes new models based on the original model $\lambda$, by changing exactly one literal at each iteration. Suppose there are k iterations in Greedy-based Algorithm, then k new models will be generated. New non-backbone are found from each new model. The heuristic strategy for Greedy-based algorithm is changing the literal that satisfies the least clauses at each iteration. When changing a literal's assignment, clauses remain their satisfiability if this literal is not their satisfied litera. Choosing the literal that satisfies the least clauses will affects the least clauses, which lead to a higher possibility of finding a new model.

\medskip
\noindent{\bf Computing an approximation of backbone.}
At this step, we apply a Whitening-based algorithm to compute the approximation $\BLap(\Phi)$.
Whitening Algorithm was used to compute \emph{essential node}, that are nodes can't be colored as white without changing the color of its adjacent nodes in a graph coloring problem.


We consider essential nodes as backbone literals in backbone computing.
To increase the proportion of backbone literals found by Whitening Algorithm, we use two heuristic strategies to refine Whitening Algorithm.
First, we check whether the generated assignment is a model to eliminate some of the non-backbone literals returned by Whitening-based Algorithm.
Moreover, we use assumptions features of MINISAT \cite{JLM15} to find some accurate backbone literals.
With the refinement of heuristic strategies, Whitening-based Algorithm is able to return a set of literals that are highly likely to be backbone.


\medskip
\noindent{\bf Computing exact backbone.}
At this step, we use Algorithm 3 (Iterative Algorithm) from \cite{JLM15} to compute backbone.
This algorithm uses SAT solvers to test whether a literal is a backbone literal or not.
For instance, if $\Phi\wedge \neg l$ is unsatisfiable but $\Phi$ is satisfiable, then $l$ must be a backbone literal of $\Phi$.

We first iteratively select one literal $l$ from $\BLap(\Phi)\setminus \NBLap(\Phi)$ such that $\lambda \models \neg l$ and test $l$ by checking the satisfiability of $\Phi\wedge l$ (dashed area in Figure \ref{flow}).
If $l$ is a backbone, we will add $l$ into $\Phi$ as a clause. Adding known backbone literals into $\Phi$ as clauses potentially speedups the later SAT testing \cite{JLM15,MPA2015}.
Then, we do the same testing for literals from $\Lit(\Phi)\setminus (\NBLap(\Phi)\cup\BLap(\Phi))$ (dotted area in Figure \ref{flow}).
After this step, the exact backbone and non-backbone are found.


In general, one can directly test all the literals to compute backbone without the two approximations.
However, the test heavily relies on SAT solving which may be time-cost.
Our approach makes two contributions compared to this na\"{\i}ve approach.
One is that we reduce the number of SAT calls using the known non-backbone literals $\BLap(\Phi)$.
The another one is that we first check literals that have high probability to be backbone literals so that backbone literals can be found as early as possible.



\section{Greedy-based Computing $\NBLap(\Phi)$}
Backbone are the common parts of all implicants. Algorithm \ref{alg:greedy} is able to compute implicants of the given formula. Compared with other implicants enumeration approach, Algorithm \ref{alg:greedy} needs less SAT testing, the returning non-backbone under approximation is able to generate several implicants. Compared with implicants enumeration \cite{JLM15}, Algorithm \ref{alg:greedy} is more efficient at blocking states, since more assignments are represented.
Experiments in \cite{JLM15} indicates that reducing SAT testing will increase performance. With more non-backbone recognized ahead without SAT testing, performance of our approach increased.

In this section, we propose a Greedy-based algorithm that computes an under-approximation of non-backbone for a given formula $\Phi$.
Although it's a rather straight forward algorithm, it is able to generate a cluster of models in $O(n^2)$.

Given a formula $\Phi$ and a model $\lambda$ of $\Phi$, let $L(\Phi,\lambda)$
denote the set $\{l\in\Lit(\Phi) \mid \lambda\models \neg l \ \mbox{or} \  \forall \phi\in\Phi, l\in\phi\Rightarrow \exists l'\in\phi\setminus\{l\}: \ \lambda\models l'\}$.


\begin{lemma} \label{lem:navie}
 $L(\Phi,\lambda)\subseteq\NBL(\Phi)$.
\end{lemma}
Intuitively, suppose $l$ is a literal in $\Lit(\Phi)$ and $\lambda$ is a model of $\Phi$.
If $\lambda$ does not satisfy $l$, i.e.,  $\lambda\models  \neg l$, then $l$ must be a non-backbone literal of $\Phi$.
Otherwise, if $\lambda$ satisfies $l$ and for all clauses $\phi\in\Phi$, $\phi$ either does not contain the literal
$l$ or contains another literal $l'$ which is also satisfied by the model $\lambda$, then it is easy to see that
the assignment $\lambda[\neg l]$ satisfies $\Phi$.
In this case, $l$ is also a non-backbone literal.

However, $L(\Phi,\lambda)$ may exclude many other non-backbone literals.
To get more non-backbone literals, we propose the Greedy-based algorithm shown in Algorithm \ref{alg:greedy}.

\begin{algorithm}
\SetKwInOut{Input}{Input}
\SetKwInOut{Output}{Output}
\SetAlgoShortEnd
\SetFillComment
\Input{a satisfiable formula $\Phi$ and a model $\lambda$ of $\Phi$}
\Output{a set of literals $\NBLap(\Phi)$}
$\NBLap(\Phi):=L(\Phi,\lambda)$\; \label{alg1:init}
%\Repeat{No Update}{\label{alg1:loopstart}
\For{${\sf HS}$ from ${\sf HS}_1$ to ${\sf HS}_2$}{
    $C:={\sf HS}(\Phi)$\;
    $i:=0$\;
    \While{$i<|C|$}{
        $i:=i+1$,  $l:=C[i]$\;
        \If{$\lambda[\neg l]\models \Phi$}
        {
           $\NBLap(\Phi):=\NBLap(\Phi)\cup L(\Phi,\lambda[\neg l])$\;
           $\lambda:=\lambda[\neg l]$\;
        }
    }
}\label{alg1:loopend}
\Return $\NBLap(\Phi)$\;
\caption{Greedy-based algorithm}
\label{alg:greedy}
\end{algorithm}
Given a satisfiable formula $\Phi$ and a model $\lambda$ of $\Phi$,
we assign $L(\Phi,\lambda)$ to $\NBLap(\Phi)$ at Line \ref{alg1:init}. Next, we update
$\NBLap(\Phi)$ using two heuristic searching strategies. % to select literalsat Lines \ref{alg1:loopstart}-\ref{alg1:loopend}.
Each heuristic searching strategy will give us an ordered set $C$ of literals, then select one literal by one literal from
$C$. For each selected literal $l$, we construct a new assignment $\lambda[\neg l]$ from the model
$\lambda$ and check whether  $\lambda[\neg l]$ satisfies $\Phi$ or not.
If $\lambda[\neg l]$ satisfies $\Phi$, then we add the set of non-backbone literals $L(\Phi,\lambda[\neg l])$ into $\NBLap(\Phi)$, and
assign $\lambda[\neg l]$ to $\lambda$ which will be severed as the model of $\Phi$ at the next step.
Obviously,  $\NBLap(\Phi)$ is an under-approximation of non-backbone of $\Phi$.
Notice that a literal $l$ such that $\lambda[\neg l]$ does not satisfy $\Phi$ may could be used
later, i.e., $\lambda'[\neg l]$ does satisfy $\Phi$ for the later new model $\lambda'$. To make the algorithm efficient,
we do not consider such literals twice in one iteration.

\begin{theorem}
$\NBLap(\Phi)\subseteq\NBL(\Phi)$.
\end{theorem}

Once, a model $\lambda$ of $\Phi$ is given, Algorithm \ref{alg:greedy} does not need to call any SAT solver which may be time-cost.
Instead, we manage to generate new models by changing assignments of literals in the known model.
We use the following two heuristic searching strategies to select literals. % with an order for generating new models.
\begin{quote}
Let $S$ be the ordered set of literals such that for every $i: 0\leq i<j<|S|$,
$f(\Phi,S[i])\leq f(\Phi,S[j])$, where $f$ is $\Cnt$ if $i=1$, and $f=\dens$ if $i=2$.
We define ${\sf HS}_i(\Phi)$ as the prefix sequence $S[0,xn]$ of $S$, for some $x\in[0,1]$.
\end{quote}

Intuitively, we choose the maximum $xn$ literals from the ordered set of literals.
The idea is that if the number $\Cnt(\Phi,l)$ of occurrence of a literal $c$ is smaller,
then the number of clauses that include the literal $l$ is smaller. This implies that the assignment $\lambda[\neg l]$ has a high probability to be a model of
$\Phi$. From the new model $\lambda[\neg l]$, we may find new non-backbone literals.
The intuition underlying ${\sf HS}_2(\Phi)$ is similar, in which we use densities of literals to sort literals instead of the numbers of occurrence.
The reason is that in longer clauses, there is a high possibility that the clauses can be satisfied by another literal. The longer the clauses are, the smaller the density of literal is. Therefore, literals with smaller density are more likely to give a new model of $\Phi$.

\section{Whitening-based Computing $\BLap(\Phi)$}

In this section, we propose an algorithm to compute approximate set of backbone literals $\BLap(\Phi)$, namely Whiten-based Algorithm. As mentioned in Section \ref{sec:overview}, finding more backbone at the earlier stage of the computing provides benefits for the later backbone computing. Experiments show that the proportions of backbone in $\BLap(\Phi)$ are generally higher than that in the original formula. Moreover, 20\% solving time is saved by Whiten-based Algorithm.

\subsection{Whitening Algorithm}

Given a formula $\Phi$, a literal $l\in\Lit(\Phi)$ is either a backbone or not. Backbone literals are essential, since an assignment $\lambda$ is not a model if there exists a backbone literal $l$, $\lambda(l)=0$. There is a similar situation in graph coloring problem, while there not exists a coloring plan if essential nodes are colored to white without changing the color of their neighbours'.

In \cite{Par03}, the authors proposed an Whitening algorithm that computed the essential nodes in a coloring problem named Whitening Algorithm. We propose a Whitening-based Algorithm for 'possible' backbone (essential) literals computing based on the original one. We use $W_c$ to denote the clauses that have at least two satisfied literals to a given model. We use $W_v$ to represent a set of variables, every variable $x$ in $W_v$ only satisfies clauses clauses in $W_c$. $W_c$ and $W_v$ are updated concurrently.

\begin{algorithm2e}
\SetKwInOut{Input}{Input}
\SetKwInOut{Output}{Output}
\SetAlgoShortEnd
\SetFillComment
\Input{a formula $\Phi$ and a model $\lambda$ of $\Phi$}
\Output{white clauses $W_c$ and white variables $W_v$}
$W_c:= \{\phi\in\Phi \mid \exists l_1,l_2\in\phi: \  \lambda\models l_1\wedge l_2\}$\; \label{alg2:c}
$W_v:=\{x\in \var(\Phi)\mid \lambda\models x\wedge x\not\in \Lit(\Psi\setminus W_c),
        \ \lambda\models \neg x\wedge \neg x\not\in \Lit(\Psi\setminus W_c)\}$\; \label{alg2:v}
\Repeat{No Update of $W_c$ and $W_v$}{ \label{alg2:loop}
   $W_c := W_c \cup \{\phi\in\Phi \mid \var(\phi)\cap W_v\neq \emptyset \}$\; \label{alg2:cadd}
   $W_v := W_v \cup \{x\in \var(\Phi)\mid \lambda\models x\wedge x\not\in \Lit(\Psi\setminus W_c),
        \ \lambda\models \neg x\wedge \neg x\not\in \Lit(\Psi\setminus W_c)\}$
}
\Return $\var(\Phi)\setminus W_v$\;
\caption{Whitening-based algorithm}
\label{alg:whitening}
\end{algorithm2e}

We first compute a set of clauses that have at least two satisfied literals under the current model, named $W_c$ at Line \ref{alg2:c}. We find variables that only satisfied clauses in $W_c$, and put them into a set of variables, named $W_v$ at Line \ref{alg2:v}. We then start to iteratively extend $W_c$ and $W_v$ from Line \ref{alg2:loop}. For every variable $v\in W_v$, if a clause contains $\neg v$, it will be added to $W_c$ at Line \ref{alg2:cadd}. After that, we compute $W_v$ again with the extended $W_c$. We repeat the procedure until no clause are added to $W_c$. At last, the complement of $W_v$ is the set of essential variables. It's important that no elements are taken away from $W_c$, the algorithm will terminate since the number of clauses is finite.

Given a model $\lambda$, the complexity of Whitening Algorithm is polynomial since it doesn't need SAT testing. For any variable $x\in W_v$, all clauses in $\Phi_{\neg x}$ will have at least two satisfied literals in the assignment $\lambda[\neg x]$, one is $\neg x$, the other is one of the satisfied literals of $\phi$ under $\lambda$. In this way, Whitening Algorithm extended $W_c$ without SAT testing.

Limited by the only model given to Whitening Algorithm, only a part of backbone variables are in $W_v$. Different models will have different $W_v$. For example, given a formula
\[(\neg a\vee b\vee\neg c)\wedge(\neg a\vee\neg b\vee c)\wedge(\neg a\vee b\vee d)\wedge(\neg c\vee d)\wedge(a\vee d)\]
and a model $\lambda$ such that $\lambda(a)=1\wedge \lambda(b)=1\wedge \lambda(c)=1\wedge \lambda(d)=1$. The result of Whitening Algorithm is empty set. It indicates that non of the variable is backbone. Actually, $d$ is a backbone literal, since there does not exist a model $\lambda$ that $\lambda(\neg d)=1$.

% At the first iteration, $\neg a\vee b\vee d$ and $a\vee d$ are selected into $W_c$, and only $a$ is selected into $W_v$. At the second iteration, $\neg a\vee b\vee \neg c$ and $\neg a\vee\neg b\vee c$ are added to $W_c$ since they contains $\neg a$. $b$ and $c$ are added to $W_v$ at next step, since it only satisfies clauses in $W_c$. All clauses are added to $W_c$ since $\neg c\vee d$ is in $W_c$ now. It ends up with all variables are added to $W_v$.

Given a model $\lambda$, the assignment change of a given variable $v$ will generate a new assignment $\lambda[\neg v]$. As we record the assignments generated during the compute step by step, we found that $a$ is a non-backbone variable, because $\lambda[\neg a]$ is another model. $b,c$ are non-backbone variables because $\lambda[\neg a,\neg b]$ and $\lambda[\neg a,\neg c]$ are models of the given formula. However, neither $\lambda[\neg a,\neg b,\neg d]$ nor $\lambda[\neg a,\neg c,\neg d]$ is a model of the given formula.

\medskip

\subsection{Assignments Checking Based Whitening Algorithm}

To avoid missing backbone literals like $d$, we propose a Whitening-check-based Algorithm (WCB for short) to compute the approximation of backbone literals with satisfiability check of each generated assignment. $\Pre(x)$ is used to record the literals that changed its assignment at each iteration. With the changed literals, we are able to generate a new assignment at each iteration. We choose a variable $x\in W_v$ at each iteration, and update $W_c$ by adding $\Phi_{\neg x}$, $W_v$ is extended accordingly. For every new variables $x'$ in $W_v$, we test the satisfiability of assignment $\lambda[\neg (\Pre(x)\cup\{x,x'\})]$. $x'$ maintains in $W_v$ if a new model is found. Otherwise, $x'$ removes from $W_v$. Algorithm stops when there is no new clause added to $W_c$.


\begin{algorithm2e}
\SetKwInOut{Input}{Input}
\SetKwInOut{Output}{Output}
\SetAlgoShortEnd
\SetFillComment
\Input{a satisfiable formula $\Phi$ and a model $\lambda$ of $\Phi$}
\Output{a set of literals $\NBLap(\Phi)$}
$W_c:=\{\phi\in\Phi\mid \exists l_1,l_2\in\phi: \lambda\models l_1\wedge l_2\}$\;\label{alg3:l1}
$W_v:=\{x\in\var(\Phi) \mid \lambda(x)=1 \forall\phi\in\Phi_x: \phi\in W_c\}$\; \label{alg3:l2}
$\forall \in W_v, \Pre(l)=\emptyset$\;
\Repeat{No update of $W_v$}{ \label{alg3:loopstart}
    \For{$x\in W_v$}{
        \For{$\phi\in\Phi_{\neg x}$}{
            $W_c:=W_c\cup\phi$\;
        }
        \For{$x'\notin W_v\wedge x'\in\Phi_{\neg x}$}{
            \If{$\forall\phi'\in\Phi_{x'}: \phi'\in W_c$}{
                \If{$\lambda[\neg (\Pre(x)\cup \{x,x'\})]\models\Phi$}{ \label{alg3:test}
                    $W_v:=W_v\cup x'$\; \label{alg3:addv}
                    $\Pre(x'):=\Pre(x)\cup x$\; \label{alg3:record}
                }
            }
        }
    }
}\label{alg3:loopend}
\Return $\var(\Phi)\setminus W_v$\;
\caption{WCB Algorithm with Assignment Satisfiability Checking}
\label{alg:ewhite}
\end{algorithm2e}

We compute $W_c$ and $W_v$ at the first two lines, and extend $W_c$ and $W_v$ at Line \ref{alg3:loopstart}. At Line \ref{alg3:test}, we test whether the generated assignment is a model of the given formula $\Phi$. At each iteration, we change the assignment of $x\in W_v$, it results in adding $x'$ to $W_v$. If the assignment $\lambda[\neg (\Pre(x)\cup\{x,x'\})]$ passes the satisfiability check at Line \ref{alg3:test}, $x'$ remains in $W_v$, and $\Pre(x')$ is $\Pre(x)\cup \{x\}$. Otherwise, $x'$ is removed from $W_v$.

The WCB Algorithm finds some missing backbone literals in Whitening Algorithm. Since the test of satisfiability at Line \ref{alg3:test} is in polynomial time, the time complexity remains the same.

\begin{theorem}
$\forall x\in W_v$, $x\in\NBLap(\Phi)$.
\end{theorem}

\begin{proof}
Given a formula $\Phi$ and a model $\lambda\models\Phi$, suppose a variable $x'\in W_v$. If $x'$ is added to $W_v$ at Line \ref{alg3:l2}, then there must exists a literal $l=x$ or $\neg l=x$, for every clause $\phi$ that contains $l$, there must exists another literal $l'$, such that $\lambda(l')=1$. Therefore, there must exists another model $\lambda'=\lambda[\neg l]$ of $\Phi$, $x$ is a non-backbone literal of $\Phi$.
If $x'$ is added to $W_v$ at Line \ref{alg3:addv}, there must exists a variable $x$, where a new model $\lambda[\neg \Pre(x)\cup \{x,x'\}]$ of $\Phi$ is generated at Line \ref{alg3:test}. Therefore, there exists two different models $\lambda_1$ and $\lambda_2$ of $\Phi$, such that $\lambda_1(x')=0$ and $\lambda_2(x')=1$. $x'$ is a non-backbone literal of $\Phi$.
\end{proof}
\medskip


\iffalse
\subsection{Dependency Graph Based Whitening Algorithm}

However, there are also some over-lapping between backbone approximation and non-backbone variables. We propose another heuristic strategy to remove parts of non-backbone literals from backbone approximation.

Consider a formula $\Phi=(a\vee\neg b)\wedge(b\vee\neg c)\wedge(c\vee\neg a)$ and a model $\lambda$ $a\wedge b\wedge c$. The backbone of the formula $\Phi$ is $\emptyset$, because the assignment $\lambda'$ such that $\lambda'(a)=\lambda'(b)=\lambda'(c)=0$ is also a model of $\Phi$.
However, by Algorithm \ref{alg:ewhite}, $W_v=\emptyset$, all variables are in backbone approximation. To remove the non-backbone variables $a, \ b, $ and $c$ from the approximation of backbone, we introduce a dependency graph to characterize such literals $a, \ b, \ c, \ \neg a, \ \neg b$ and  $\neg c$ from  the given formula $\Phi$.

Given a formula $\Phi$, the \emph{dependency graph} $G_\Phi$ of $\Phi$ is a directed graph $(V,E)$, where
$V=\Lit(\Phi)$ is a finite set of vertices, and $E\subseteq V\times \Phi\times V$ is a finite set of labeled edges such that
$(l_1,\phi, l_2)\in E$ iff $l_1,\neg l_2\in \phi$.

Figure \ref{fig:depend} depicts the dependency graph of the formula $\Phi=\phi_1\wedge\phi_2\wedge\phi_2$, where
$\phi_1=(a\vee\neg b)$, $\phi_2=(b\vee\neg c)$ and $\phi_3=(c\vee\neg a)$.
 \begin{figure}
    \centering
    \includegraphics[scale=0.7]{dependency.pdf}
   \caption{Dependency graph of $(a\vee\neg b)\wedge(b\vee\neg c)\wedge(c\vee\neg a)$}
   \label{fig:depend}
\end{figure}

A \emph{path} $\pi$ of a dependency graph $G_\Phi=(V,E)$ is a sequence $l_1...l_n$ of nodes such that for every $i:\ 1\leq i<n$, $(l_i,\phi_i,l_{i+1})\in E$.

A \emph{simple cycle} $c$ of $G_\Phi$ is a path of $G_\Phi$ such that the starting and ending nodes are identical and no repetitions of other nodes.
We use $G_\Phi^{c}$ to denote the set of all the simple cycles of $G_\Phi$.
Given a set of cycles $C\subseteq G_\Phi^c$,
Let $\var(C)$ (resp. $\Lit(C)$ and $E(C)$) denote the set of variables (resp. literals and edges) appeared in $C$,
and $\Phi_{C}$ denote the set of clauses labeled to the edges in $C$.

\begin{proposition}\label{prop:scc-path}
Given a formula $\Phi$ and a set of simple cycles $C\subseteq G_\Phi^{c}$,
if there is no shared edges between each pair of simple cycles in $C$,
then the formula $\Phi_{C}$ is satisfiable. Moveover, for every model $\lambda$ of $\Phi_{C}$,
the assignment $\lambda[\neg \var_{C}]$ is a model $\Phi_C$.
\end{proposition}

\begin{proof}
Consider the assignment $\lambda$ of $\Phi_{C}$ and a cycle $c\in C$, such that for each pair of edges $(l_1,\phi,l_2),(l_1',\phi,l_2')$ of $E(\{c\})$,
$\lambda\models l_1$ iff $\lambda\models l_1'$, and $\lambda\models l_2$ iff $\lambda\models l_2'$.
Then, $\lambda$ is a model of $\Phi_{\{c\}}$. The result immediately follows.
\end{proof}

\begin{lemma}
Given a satisfiable formula $\Phi$ and a set of simple cycles $C\subseteq G_\Phi^{c}$, if $\var(C)\cap \var(\Phi\setminus\Phi_{C})=\emptyset$ or $\Lit(C)\cap \Lit(\Phi\setminus\Phi_C)=\{l\}$, where the literal $l$ is a non-backbone literal for the formula $\Phi\setminus\Phi_C$, then the literals in $\Lit(C)$ are non-backbone literals.
\end{lemma}

\begin{proof}
If $\var(C)\cap \var(\Phi\setminus\Phi_{C})=\emptyset$, then for every $x\in\var(C)$, $\Phi\setminus\Phi_C$ does not contain any
literal of the form $x$ or $\neg x$. Let $\lambda$ be a model of $\Phi$.
By Proposition \ref{prop:scc-path}, $\lambda[\neg\var(C)]\models\Phi_C$.
Therefore, $\lambda[\neg\var_{scc}(\Phi) ]\models\Phi\setminus\Phi_C$.
The result immediately follows.

If $\Lit(C)\cap \Lit(\Phi\setminus\Phi_C)=\{l\}$, where the literal $l$ is a non-backbone literal for the formula $\Phi\setminus\Phi_C$,
there exist two models $\lambda_0$ and $\lambda_1$ of $\Phi$ such that $\lambda_0(x)\neq \lambda_1(x)$.
Let $\lambda_i'$ for $i\in\{0,1\}$ be the assignment such that
for every $y\in\var(C)$, $\lambda_i'(y)=i$, and for every $y'\in \var(\Phi)\setminus\var(C)$,
$\lambda_i'(y')=\lambda_j(y')$ for some $j\in\{0,1\}$ with $\lambda_j(x)=\lambda_i'(x)$.
Obviously, $\lambda_0'$ and $\lambda_1'$ are two models of $\Phi$. Therefore,
the literals in $\Lit(C)$ are non-backbone literals.
\end{proof}

We adapt the algorithm from \cite{Jon75} to identify non-backbone literals from $F(\Phi,\lambda)$ and these non-backbone literals are added in $\NBLap(\Phi)$.
This gives a more tight approximation set of backbone literals which is the desired set $\BLap(\Phi)$.
\fi

\subsection{Computing Parts of Exact Backbone Using Assumptions}

Although WCB Algorithm is able to compute an approximation set of backbone, it still needs at least one SAT testing to determine whether a literal is backbone, which may need a long solving time. Inspired by \cite{JLM15}, we use assumptions in MINISAT as a heuristic strategy to accelerate SAT testing, named Whitening-assumptions-based Algorithm, WAB for short. 

Experiments show that given the same formula $\Phi$, SAT testings with assumptions are generally faster then the ones without assumptions. It's because that assumptions help to make the initial decisions of a SAT solving, given an assumption $\gamma$ with k literals in it, it reduces $2^k$ states of the searching spaces. SAT testing with a longer assumptions will return faster.

A formula $\Phi$ is satisfiable with assumption $\gamma$ indicates that there exists a model $\lambda\models\Phi$, such that $\forall l\in\gamma, \lambda(l)=1$. We use $SAT(\Phi,\gamma)$ to denote a SAT testing of $\Phi$ with the assumption of $\gamma$. We use $(b, \lambda, r)$ to denote the result of $SAT(\Phi,\gamma)$. If $b$ is assigned to $1$, a model is returned in $\lambda$, otherwise, a reason of unsatisfiable is returned in $r$. For every literal $l\in\BLap(\Phi)$, $\neg l$ is selected to $\gamma$.
If there is exactly one reason $l_b$ returned from $SAT(\Phi,\gamma)$, it indicates that $\neg l_b$ must be a backbone of $\Phi$.


\begin{algorithm2e}
\SetKwInOut{Input}{Input}
\SetKwInOut{Output}{Output}
\SetAlgoShortEnd
\SetFillComment
\Input{a satisfiable formula $\Phi$ and $\BLap(\Phi)$}
\Output{under-approximation of backbone literals $\BLapu(\Phi)$ }
$\BLapu(\Phi):=\emptyset$\;
$\gamma:=\{l\in\Lit(\Phi) \mid \neg l\in\BLap(\Phi)\}$\; \label{alg4:init}
\Repeat {$\gamma==\emptyset$}{
    $(b, \lambda', r):=SAT(\Phi,\gamma)$\; \label{alg4:test}
    \If {$b==0$}{
        \If {$|r|==1$}{
            $l_b:=r_0$\;
            $\BLapu(\Phi):=\BLapu(\Phi)\cup \{\neg l_b\}$\; \label{alg4:bl}
        }
        $\gamma=\gamma\setminus r$\; \label{alg4:cut}
    }
    \If{$b==1$}{
        $\BLap(\Phi):=\BLap(\Phi)\setminus\gamma$\; \label{alg4:remove}
        $\NBLap(\Phi):=\NBLap(\Phi)\cup\gamma$\;
        break\;
    }
}
\Return $\BLapu(\Phi)$\;
\caption{WAB Algorithm for computing under-approximation of backbone $\BLapu(\Phi)$}
\label{alg:assum}
\end{algorithm2e}

Given a model $\lambda$ and a satisfiable formula $\Phi$. We initialize $\gamma$ with $\BLap(\Phi)$ at Line \ref{alg4:init}. We add $\neg l$ to $\gamma$ for every $l\in\BLap(\Phi)$ to block the known model $\lambda$. At Line \ref{alg4:test}, we compute the result of $SAT(\Phi,\gamma)$. All literals in $\gamma$ will be removed from $\BLap(\Phi)$ at Line \ref{alg4:remove} if $b$ is assigned to $1$. Otherwise, all literals in $r$ are removed from $\gamma$ at Line \ref{alg4:cut}. Once $\gamma$ is empty, the iteration stops. If there is exactly one literal $l_b$ in $r$, $\neg l_b$ must be backbone and added to $\BLapu(\Phi)$ at Line \ref{alg4:bl}.

\begin{theorem}
Given a satisfiable formula $\Phi$ and a set of assumptions $\gamma$, $\neg l_b\in\BL(\Phi)$ if $l_b\in\gamma$ and $l_b$ is the only reason that $\Phi$ is not satisfiable under the assumption of $\gamma$.
\end{theorem}

\begin{proof}
Given a satisfiable formula $\Phi$ and a set of assumptions $\gamma$, suppose literal $l_b$ is the only reason returned $r$ from $SAT(\Phi,\gamma)$, i.e., $\forall l\in r, l=l_b$.
It means that there doesn't exist a model $\lambda\models\Phi$, such that $\lambda(l_b)=1$. Therefore, $\neg l_b$ is a backbone literal of $\Phi$, i.e., $\neg l\in\BL(\Phi)$.
\end{proof}

WAB Algorithm is able to compute the approximation of backbones of the given formula. When the length of unsatisfiable reason is 1, WAB saved solving time when determining if a variable is a backbone. Experiments show that, SAT testings with assumptions are generally finished within 1 second, while original SAT testing may take more than 1 minute. 

Compared with the original Whitening Algorithm, the accuracy of Whitening-based Algorithm improved by two heuristic strategies (WCB and WAB) has increased. Missing backbone are added to the approximation through WCB and accurate backbone are confirmed with WAB. With a higher accuracy, backbone computing are guided better with $\BLap(\Phi)$.

\section{Experimental Study}\label{sec:expr}
We implemented our approach in a tool called \tool written in C++ interfacing MINISAT 2.2 \cite{MINISAT}.
It's available at xxx.
In this section, we present the experimental results of \tool.
There are 3 industrial benchmarks, consists of 72 industrial satisfiable formulae and 86 random benchmark, consisted 6606 generated satisfiable formulae. Each formula in an industrial benchmark is selected from verification of the same application, each formula in a random benchmark are generated from the same unsatisfiable formula. Formulae in the same benchmark share some common features, such as the number of variables and clauses, the structure of the formula and so on.

The experiments were conducted on a cluster of IBM iDataPlex 2.83 GHz, each industrial formula was running with a timeout of 1800s and memory limit of 4GB. Each random formula was running with a timeout of 100s and memory limit of 256 MB. Our experiments benefit from the small scale of generated formulae, which allow us to run more formulae at the same time.

For industrial formulae, both \tool and \minibones are able to solve 34 formulae out of 72. For generated formulae, both \tool and \minibones are able to solve the total 6606 formulae under time and memory limits.
Experiments show that for \textit{mrpp} benchmark, \minibones solved 4\% times faster, and for $\textit{manthey}$ benchmark, \tool reduces 38\% solving time. \tool reduces 2183 seconds of solving time of industrial formulae, saving 21\% solving time in total.

When we take a close look at the formulae that are accelerated by \tool from industrial benchmark, we found that the adjacent structure of these formulae shared some common features. Experiments show that different formula has different structures, formulae from the same benchmark often have similar structures. Compared with the ball-like structure for formulae in $\textit{mrpp}$, formulae in $\textit{manthey}$ benchmark have more star-like structures. There are two formulae in $\textit{mrpp}$ that have a pipe-like structure, where there are several branches near the main root-like core, \tool performs better on this two formula.
\subsection{Benchmark Setup}
We select 72 industrial formulae from SAT competitions, 34 are them are solved by both \tool and \minibones. We also select 100 crafted formulae from SAT competitions, none of them are solved within time and memory limits. In order to test the scalability of \tool, we generate 6606 satisfiable formulae from unsatisfiable formulae. The reason we don't choose random satisfiable formulae is because they have different community structures. Generated formulae from the same unsatisfiable formula share similar community structure from nature. We would like to explore the relationship between features of structures of the formula and the performance of \tool. Another reason is that, unlike other random formula that requires more than 500 seconds solving time, the solving time of generated formula range from 1 seconds to 100 seconds, which saved lots of solving time and make our experiments more practicable.

There are 3 benchmarks among industrial formulae, which are $\textit{mrpp}$, $\textit{manthey}$ and $\textit{dimacs}$. The comparison between \tool and \minibones are analyzed in different benchmarks.
For generated formulae, they are grouped into 100 benchmark since there are 100 different formulae that generate them.

We separate formulae into different benchmarks to study the effects of different adjacent structures on the performance of \tool. Adjacent structures are presented as adjacent graph of a formula. The nodes in the graph stands for variables, if two variables shared the same clause, there is an edge between them. In the graph, the more adjacent variables a variable has, the more center the variable located. In the middle of an adjacent graph, there is a ball-like core formed by variables that has multiple neighbours, if a variable has less neighbours, it will locate apart from the core, the less neighbours a variable has, the farer it located to the core.

\subsection{Experimental Results on Industrial Formulae}
The general result of industrial formulae
SAT solving time and SAT calls of different benchmark, with different proportion.
A line graph that describes SAT solving time comparison.

Show the density of $\BLap(\Phi)$ and the original formula, show them in line graph, they should be always higher.

Analysis the relationship between adjacent structure and the accelerating performance, give two structures as example. The ball-like one are hard for both \tool and \minibones. But \tool performs better on branch-like ones, thanks to the remove of non-backbone ahead and the high possible of backbone, all increase the chance that we hit one of the variables in the branch, and SAT solving will be rather fast. Since variables are less connected with each other and less conflicts will arise.


SAT solving time and SAT calls of different benchmark, only with Algorithm 1.
The same table and analysis for Algorithm 2. a table with 1, 2, 1+2.
A general comparison result about Algorithm 1 and Algorithm 2.
The results show that for formulae in $\textit{manthey}$, the combination is the best. The contribution is almost the same.
For $\textit{mrpp}$, the combination is not the best solution, this explains why our algorithm performs poorly on this benchmark, one step of the computing doesn't behaves well. It's because that the formula in this benchmark are hard to test whether a literal is a backbone,(seen the gap between the first model solving time and the total solving time). The reason for this is because the variables here shared a very tight relations, it will causes lots of conflicts by mis-assignement any of them.


\begin{table}[t]
\centering
\begin{tabular}{ccccc}
\toprule
 &Total  Time (s) & SAT Calls&Average Time (s)\\
\midrule
\tool&26663  &272089&0.9890  \\
cb100&30295  &239112&1.0174  \\
\bottomrule
\end{tabular}
\caption{Experimental results for 138 industrial formulae}
\label{tab:ind}
\end{table}

Table \ref{tab:ind} shows the overview of experimental results on industrial formulae.
Among all the 388 industrial formulae, there are 138 formulae that both \tool and \textit{cb100} were able to compute backbone within 1800s.
Although, the number of SAT calls used by \tool is greater than \textit{cb100}, \tool reduces by 12\% of the total running time and by 11\% of the average running time for each SAT call.

Figure \ref{fig:ind-time} provides the details of experimental results on industrial formulae, including the running time of the SAT solver for generating the first model and total running time for each formula.
The red dotted line represents for the time used by the SAT solver for computing the first model, which indicates the hardness of the formula.
Lines with crosses represents for the running time of \tool, and lines with boxes represents for the running time of \textit{cb100}. There is a correlation between the hardness of formulae the performance of \tool comparing to \textit{cb100}.
For most of hard formulae, \tool outperforms \textit{cb100} in terms of total running time and average time of each SAT solver call. This demonstrates that \tool is more feasible for computing backbone of hard formulae.

%\textit{cb100} highly rely on the quality of model returned by SAT solver, it performances better when SAT solver gave it a suitable model. For most of the formulae that need less than 1 second to compute a first model, both \tool and \textit{cb100} were able to compute backbone very quickly with barely difference.

Table \ref{tab:ind} and Figure \ref{fig:ind-time} demonstrate that \tool outperforms \textit{cb100} on industrial formulae.


\begin{figure}
    \centering
    \includegraphics[scale=0.8]{ind2.pdf}
   \caption{Experimental results on industrial formulae}
   \label{fig:ind-time}
\end{figure}



\section{Results for Generated Formulae}
How many solving time saved in total, what proportion.
Show the total solving time and SAT calls of selected 3 benchmark, accelerating 1, the same 1, decline 1.
Show the statistic results of different benchmarks, how many accelerating, more than how large proportion, how many the same, how many decline, at how large proportion

Explain the relation between adjacent structure and performance.
3 pictures

Based on this observation, we divide 6606 random formulae into three groups according to their community structures.
The \emph{simple group} contains 371 formulae whose community structures are divergent,
the \emph{hard group} contains 337 formulae whose community structures are focusing,
and the \emph{medium} group contains 5898 formulae between divergent and focusing.


Table \ref{tab:mcs-graph} shows the experiments results on these three groups. \textit{cb100} performs better than \tool on the simple group, as \textit{cb100} tries to find a backbone literal by complementing models as soon as possible, which are the vertexes far from the center.
\tool outperforms \textit{cb100} for both medium group and hard group.
\tool reduces by 8\% for medium group and by 40\% for hard group of the total running time.

The experimental results show that \tool performs better than \textit{cb100} on hard random formulae and is comparable to \textit{cb100} on simple formulae.

NOTE: draw the community structure of the unsatisfiable formula
For formulae generated from an unsatisfiable formula with a very sharp community structure, that is there is exactly one obvious edge that sperate away from the core of variables, \tool performs quite well, almost needs less than 5 seconds to compute backbone.
For formulae generated from an unsatisfiable formula with a ball-like structure, both \tool and \minibones need almost more than 20 seconds to solve. This indicates that ball-like structure are hard to complex backbone. If we compare the solving time of first model between type I and type II, we can find that it's more changeling for MINISAT to solve type II, because there is no short-cut to find a good assignment quickly.
For the rest of formula, a core with multiple branches around the core, \tool performs better than \minibones on most of the generated formulae.





\section{Related Work}\label{sec:relw}
% This paper is concerned with computing backbones of propositional formulae, which was oriented from coloring problem \cite{CJG2001}, with a wide range of practical applications such as MaxSAT \cite{MMBM2005}.

%A number of backbones extraction algorithms have been proposed in recent years.
Kaiser and K\"{u}hlin proposed three model enumeration based algorithms for computing backbones \cite{KKW2001} using SAT solvers.
%The first one iteratively assigns true (false) to each variable and tests the resulting formula for satisfiability.
%The second one reuse the results of previous satisfiability checks and the last one maximizes the number of variables that
%an be classified without satisfiability checking, which share same purpose as our work.
%
Dubois and Dequen proposed a heuristic search for computing backbones of hard 3-SAT formulae which yields DPL-type algorithms with a significant
performance improvement over the best previous algorithms \cite{DD2001}.
Climer et al. proposed a graph-based approach to discover backbones which approximates lower and upper bounds to compute
backbones \cite{CZ2002}.% of instances of the travelling salesman problem .
%Kilby et al. showed that backbones are hard even to approximate and proposed algorithms for computing backdoors which little overlap with backbones \cite{KPS2005}.
%and proposed algorithms for computing backdoors, that are literals/variables whose absence will simplify formulas to be solved in polynomial time %\cite{KPS2005}. As discussed by \cite{KPS2005}, backbones have little overlap with backdoors.
%
%
Zhu et al, proposed an iterative SAT testing based algorithm \cite{ZWSM11,ZWM11} which is more efficient than previous model enumeration.
%At the first step, this algorithm assigned the first model returned from SAT solver to backbone estimation, which need less memory to compute backbone.

Marques-Silva et al.  investigated algorithms for computing backbones emphasizing the integration of existing algorithms which include model enumeration, iterative SAT-testing and filtering with modern SAT solvers, as well as optimisations.
They conducted an experimental evaluation of existing techniques and showed that backbone computation for large practical formulae is feasible. \cite{MJML2010,JLMS12,JLM15}.
%In this work, we proposed a novel Greedy-Whitening based approach \tool.
%Experimental results demonstrated that our approach performs better than the best one of algorithms in \cite{MJML2010,JLMS12,JLM15} for industrial formulae and
%hard random formulae, while for simple formulae, \tool is also comparable.





%The algorithm maintained an estimation of backbone. In each iteration, a clause that formed by the negation of backbone estimation is added to $\Phi$. If $\Phi$ was satisfiable, it implied that at least one non-backbone literal is in the backbone estimation. Intersection between the model given by SAT solver and the backbone estimation indicates the non-backbone literal. In this way, non-backbone literal is removed, the process is repeated until $\Phi'$ is not satisfiable any longer. Along with the estimation, the clauses number of $\Phi$  is monotone increasing due to the continuously insertion in each iteration, which dramatically promote the complexity of $\Phi$. In other words, for each iteration, it takes longer CPU time than the last iteration.

%Janota et al, proposed an Iterative algorithm (one test per variable) \cite{JLM15}. For each iteration, it added only one unit clause to the original formula, which made the new formula easier. Inspired by this algorithm, our approach in this paper follows this idea.

%The Core Based Algorithm presented in \cite{JLM15} is stable and effectiveness. The cb100 tool, as our main comparative object, was based on this algorithm. %Instead of adding only one unit clause to the original formula, this algorithm added all the unit clause to the formula. It will dramatically accelerate SAT solving. Although there is a high possibility that the new formula is unsatisfiable, whenever there was only one literal in unsatisfiable reason of the new formula, this literal is a backbone literal. In this way, it was able to find backbone literals with little cost. In this paper, the author showed that for formulae with backbone percentages lower than 25\%, Core Based Algorithms are better.
%significantly better. When the percentage of backbone is over 25\%, Core Based Algorithm behave very similarly Iterative SAT Testing Algorithm.


%Kilby et al proposed \cite{KST2005} that there was little overlap between backbone and backdoors.

%In \cite{KKW2001}, they partitioned backbone into inadmissible and necessary group. For inadmissible group, literals were false in each satisfying variable assignment. For necessary group, literals were true in each satisfying variable assignment. They described and compared three algorithms for searching the set of necessary and inadmissible variables, a basic iterative testing and two enhancements. The first enhancement was reusing the result of satisfiability checks to get more models for free, the second enhancement was selecting the decision variables for SAT solvers according to the previous information of inadmissible and necessary group.


%In \cite{DOG2001}, they proposed a heuristic search for backbone, and choose this backbone variables as branch nodes for the tree developed by a DPL-type procedure. Experiments showed that a significant performance improvement over the best current algorithms, and enhanced the scalability of the algorithm up to 700 variables.

%In \cite{WS2001}, they concluded the correlation of backbone size with the problem of optimization and approximation, using graph coloring, Travel Salesman, Number partition and block words planning. And they suggested that it is necessary to eliminate trivial cases of backbone before using backbone size to evaluate the hardness of optimization and approximation problems.


\section{Conclusions}\label{sec:conc}
%This paper proposes a novel greedy-whitening based approach(\tool, for short) to compute backbone of a propositional formulae.
%\tool first computes an under-approximation of non-backbone, which can save SAT solving counts and running time of the formula.
%Then, based on the under-approximation of non-backbone, \tool computes the over-approximation of backbone, which helps \tool to find a backbone literal %earlier.
%Finally, \tool iteratively tests literals to see if they are backbone literals, which is inspired by Iterative Algorithm.

%The experimental results show our approach is efficient, especially for industrial formulae which need longer time to compute the first model.
%Future improvements to backbone computation algorithms include parallel approximations, automatically identification of partitions and more accurate community %structure analysis.


%\section{Conclusion}\label{sec:conc}
In this paper, we proposed a backbone computing approach named \tool, using Greedy-Whitening based algorithm.
First, we computed an under-approximation of non-backbone $\NBLap(\Phi)$ using Greedy-based Algorithm in $O(n^2)$ time. Literals in $\NBLap(\Phi)$ didn't need an iterative SAT testing, the reducing number of SAT testings resulted in the saving of solving time.
Next, we computed an approximation of backbone $\BLap(\Phi)$ using Whitening-based Algorithm. We iteratively extended the set of clauses $W_c$ and variables $W_v$ accordingly. $\BLap(\Phi)$ was the complement of $W_v$. Experiments showed that the proportions of backbone in $\BLap(\Phi)$ were higher than that in the original formula. Finding more backbone earlier will expedite backbone computing. It's because that more known backbone can prune more states in SAT solving.
At last, we iteratively determined whether a literal is backbone from previous approximations.

We compared \tool with state-of-art tool \minibones, on both industrial formulae and random formulae.
Experimental results for industrial formulae demonstrated that both \tool and \minibones were able to solve 34 formulae from 72 formulae in 3600 seconds.
\tool solved 49 formulae while \minibones solved 47 formula when the time limit was 16000 seconds.
It indicated that \tool performs better than \minibones when the computing needed more time.
When the time limit was 3600 seconds, we ran experiments on three industrial benchmark selected from SAT competitions. \tool saved 21\% solving time in total than \minibones does.

\tool performed better than tool \minibones on $\textit{manthey}$  benchmark selected from SAT competitions of industrial track. For every formula in $\textit{manthey}$ \tool needed less solving time than \minibones does, 34\% of solving time was saved by \tool.
For the other two industrial benchmarks, \tool outperformed \minibones with less 8\% solving time on $\textit{mrpp}$, and less 11\% solving time on $\textit{dimacs}$ benchmark.
For random formulae with 250 variables and 1065 clauses, \tool saved 1\% solving time compared to \minibones.
Experiments on 6600 formulae indicated that \tool outperforms \minibones if more than 80\% variables have over 10 adjacent variables.

Empirical results indicated that \tool performs better than \minibones on more complex adjacent structure. Industrial formulae have more complex adjacent structures than random formulae, since the scale of industrial formulae are larger. \tool performs better on industrial formulae than random formulae.

There were two major strategies used in Greedy-Whitening Algorithm, experiments showed that they performed differently on different benchmarks when applied independently, it opens a possibility for portfolio approach. How to decide which strategy to use on a given benchmark is the most important part portfolio approach.



%\input {algorithm_unique_satisfy.tex}


\newpage

\bibliography{bib}

\begin{thebibliography}{99}

\bibitem{r1}
\textit{Scientific Style and Format: The CBE manual for authors,
editors and publishers}. Style Manual Committee, Council of Biology Editors.
Sixth ed. Cambridge University Press, 1994.

\bibitem{r2}
L.U. Ante, Cem surgere: Surgite postquam sederitis, qui manducatis panem doloris,
\textit{Omnes} \textbf{13} (1916), 114--119.

\bibitem{r3}
T.X. Confortavit, \textit{Seras}, Portarum, New York, 1995.

\bibitem{r4}
P.A. Deus, Ater hoc et filius et mater praestet nobis,
\textit{Paterhoc} \textbf{66} (1993), 856--890.

\end{thebibliography}
\end{document}
