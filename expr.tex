\section{Experimental Study}\label{sec:expr}
To study the performance of our tool \tool, we compare it to state-of-art tool \minibones and analysis total solving time for different groups of formulae.
We implemented \tool in C++ interfacing MINISAT 2.2 \cite{MINISAT}.The experiments were conducted on a cluster of IBM iDataPlex 2.83 GHz, each industrial formula was running with a memory limit of 4GB. Each random formula was running with a memory limit of 256 MB.

In the experiments, we separate formulae into groups. For example, if a set of formulae are generated from the application of hardware model checking, then they belong to the same group. We study the performance of \tool in different groups, results show that \tool saved 21\% solving time in total and performs the best in $\textit{manthey}$ group, saving 34\% of solving time.
We also conduct experiments on 86 groups of random formulae, which are generated from 86 unsatisfiable formulae.
\tool saved 0.4\% solving time among all random formulae. In all 86 groups, \tool performs faster than \minibones on 58 of them.

Experiments show that both \tool and \minibones are good at solving industrial formulae which have clear partitions of variables, \tool performs better than \minibones in general. Compared to industrial formulae, random formulae have more clear variables partitions, and are less complex than industrial formulae. In the experiments of random formulae, results indicate that \tool is good at solving formula with more variables that have over 10 adjacent variables. More adjacent variables make the formula more complex. Together with two experiments, we can conclude that for formulae that have variables partitions, \tool performs better than \minibones if the structures of the formulae are more complex.

\subsection{Benchmark Setup}

% It's available at xxx.
We selected 72 industrial formulae and 100 crafted formulae from SAT competitions between 2002 and 2016. Results show that non of the crafted formula is solved by any tool. Therefore, instead of using crafted formulae, we generate 6600 satisfiable formulae from 86 unsatisfiable formulae, selected from Uniform-3-SAT problem. The formulae generated from the same unsatisfiable formula shared the same features and all 6600 formulae have the same variables and clauses number. We consider it fair using random formulae to test the scalability of \tool and \minibones, since both of them applied MINISAT as its SAT solver.
There are 3 groups among industrial formulae, which are $\textit{mrpp}$, $\textit{manthey}$ and $\textit{dimacs}$. And 86 groups among 6600 random formulae, separated by the original unsatisfiability naturally.

\subsection{Means of Presentation}
The experiments consists of two parts, the first part shows the results in industrial formulae and the second part shows results among random formulae.
We use $\textbf{st}$ to denote the solving time and $\textbf{sc}$ to represent SAT testings number.
A number of results are shown as plots figures. In the figures, each line represents a tool performance of the input formulae. The x-axis stands for different formulae and the y-axis stands for the solving time of the corresponding formulae.

\subsection{Experimental Results on Industrial Formulae}\label{sec:ind_expr}
Among the 72 industrial formulae, both \tool and \minibones are able to solve 34 of them in 3600 seconds (1 hour).
If more solving time is provided, \tool solved 49 formulae in 16000 seconds (almost 5 hours) while \minibones solved 47 formulae.
The details of solving time and SAT tests number comparison of each benchmark are shown in Table \ref{tab:ind} (with 1 hour time limits). We can observe that \tool performs the best on $\textit{manthey}$ group, saved 34\% solving time compared to \minibones, with 14\% saving on SAT tests numbers. In total, \tool saved 21\% solving time and 16\% less SAT calls. Our experiments result prove the observation in \cite{JLM15} that less SAT calls will lead to a faster solving.
Figure \ref{fig:ind} shows the solving time of all 34 industrial formulae, there are only 3 formulae that \minibones outperforms \tool.

When we take a look at the formulae in $\textit{manthey}$ group that \tool performs the best, we find that they all have a star-like adjacent structure as shown in Figure \ref{fig:cs} (a). It means that the variables is separate into different partitions, different variables from different partitions only appears in a few clauses at the same time. Each branch in the graph stands for a partition.



\begin{table}[t]
\centering
\begin{tabular}{ccccccc}
\toprule
 Benchmark &$\textbf{st}$ of \tool(s) &$\textbf{st}$ of \minibones (s) & $\textbf{st}$ Difference &$\textbf{sc}$ of \tool &$\textbf{sc}$ of \minibones \\
\midrule
mrpp & 6112 & 6900 & \textbf{11\%} & 19541 & 22839 \\
manthey & 4845 & 7363 & \textbf{34\%} & 49356 & 59562 \\
dimacs & 1339 & 1369 & \textbf{2\%} & 2018 & 2022 \\
total & 12296 & 15632 & \textbf{21\%} & 70915 & 84423 \\
\bottomrule
\end{tabular}
\caption{Solving Time and SAT Tests Number Comparison on Industrial Formulae}
\label{tab:ind}
\end{table}

\begin{figure}
    \centering
    \includegraphics[scale=0.5]{ind.png}
   \caption{Solving Time Comparison on Industrial Formulae}
   \label{fig:ind}
\end{figure}

Figure \ref{fig:cs}(b) is the adjacent structure of formulae that can't be solved in 16000 seconds using both tools. It's obvious that there is no clear partition among the variables.


\begin{figure}[t]
\centering
\subfloat[Formulae with clear variables partition]{

\centering
\includegraphics[width=.5\linewidth-0.45mm]{star.png}
}
\subfloat[Formulae with a tightly co-related variables]{
\centering
\includegraphics[width=.5\linewidth-0.45mm]{ball.png}
}
\caption{Adjacent structures}
\begin{minipage}[t]{0.9\textwidth} \small
\textit{(Figure (a) shows the star-like adjacent structure, indicates that variables separate into individual partitions, both \tool and \minibones are able to solve these formulae in 3600 seconds, \tool saved 34\% solving time in total. \\
Figure (b) shows the adjacent structure with a core and no branches, indicates that variables are tightly co-related in the formula. These formulae are more complex than the formula in \ref{fig:cs}(a), \tool solved 2 more formulae than \minibones among these formulae, within 16000 seconds)}
\end{minipage}

\label{fig:cs}
\end{figure}

For industrial formulae, \tool performs better than \minibones in general. We also found that the performance of \tool and \minibones are related to the adjacent structure of the given formula. A formula with clear partitions of variables tend to performs better on both \tool and \minibones.
To analysis the relations between formula structure and the performance of both tools, we designed and implemented experiments with 86 groups of generated formulae, consists of 6600 formulae in total. Every formula shares the same structure feature with other formula in the same group. All formulae shared the same number of variables and clauses.

\subsection{Results for Random Formulae}
Among all the 6600 random formulae, both \tool and \minibones are able to solve them within 50 seconds. Table \ref{tab:mcs_all} shows the total solving time and total SAT testings number of \tool and \minibones. 0.4\% solving time and 0.1\% SAT testings are saved by \tool.

\begin{table}[t]
\centering
\begin{tabular}{ccc}
\toprule
  &$\textbf{Solving Time}$ & $\textbf{SAT Testings Number}$ \\
\midrule
\tool & 50184 & 1508723  \\
\minibones & 50414 & 1532372 \\

\bottomrule
\end{tabular}
\caption{Solving Time and SAT Tests Number Comparison on Random Formulae}
\label{tab:mcs_all}
\end{table}

We separate the formulae into 86 groups according to the original unsatisfiable formula. \tool needs less solving time among 58 of them. Inherited from the industrial experiment, we consider the adjacent structures of different groups. Among all the 6600 generated formulae, \tool performs better for most of the case if there are more than 210 variables that have more than 10 adjacent variables. We refer variables with more than 10 adjacent variables as $\textit{active variables}$ since they appears in many clauses. It's obvious that with more active variables, the formula is more complex. We randomly select 10 formulae groups and demonstrate the average active variables number and total solving time of both tools in Table \ref{tab:mcs}. We can observe that most of the formulae performs better on \tool have more than 210 active variables.

\begin{table}[t]
\centering
\begin{tabular}{ccccc}
\toprule
Active Variables Number & $\textbf{st}$ of \tool(s) & $\textbf{st}$ of \minibones(s) & $\textbf{st}$ Difference\\
\midrule
203 & 354 & 346 & -2\% \\
208 & 614 & 574 & -6\% \\
215 & 346 & 400 & \textbf{13\%} \\
211 & 854 & 903 & \textbf{5\%} \\
215 & 365 & 381 & \textbf{4\%} \\
201 & 475 & 469 & -2\% \\
216 & 626 & 669 & \textbf{6\%} \\
204 & 643 & 592 & -7\% \\
208 & 389 & 424 & \textbf{8\%} \\
214 & 580 & 616 & \textbf{6\%} \\

\bottomrule
\end{tabular}
\caption{Active Variables Number and Solving Time of 10 Selected Random Formuale}
\label{tab:mcs}
\end{table}



